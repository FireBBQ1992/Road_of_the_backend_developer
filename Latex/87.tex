\usepackage{amsmath,amssymb,gensymb,bm,fancyhdr,braket,mathrsfs}
\usepackage{multirow,multicol,ifthen,array,harpoon}
\usepackage{systeme,fancybox,times}
\usepackage{indentfirst,verbatim,subfigure,slashbox}  
\usepackage{url,booktabs,setspace,eso-pic,cellspace,scalerel}
\usepackage{makecell}
\usepackage{setspace}   % Set space between lines.
\usepackage{xspace}     % Define commands that appear not to eat spaces.
\usepackage{fontspec} %加這個就可以設定字體
\usepackage{xeCJK} %讓中英文字體分開設置
\setCJKmainfont{NotoSansTC-Regular.ttf}
\XeTeXlinebreaklocale "zh" 
\XeTeXlinebreakskip = 0pt plus 1pt
\usepackage{enumerate} %選擇題
\usepackage{amsfonts,amsthm,textcomp,tabularx,accents,extpfeil}
\usepackage{color} 
\usepackage{hyperref,afterpage}
\usepackage{framed}
\usepackage{enumitem}
\usepackage{varwidth}
\usepackage{hhline}%繪製特殊表格線
\usepackage{colortbl}%製作彩色表格
\usepackage{xcolor}%定義表格顏色
\usepackage{empheq}
\usepackage{graphicx}
	\definecolor{hellmagenta}{rgb}{1,0.75,0.9}
	\definecolor{hellcyan}{rgb}{0.75,1,0.9}
	\definecolor{hellgelb}{rgb}{1,1,0.8}
	\definecolor{colKeys}{rgb}{0,0,1}
	\definecolor{colIdentifier}{rgb}{0,0,0}
	\definecolor{colComments}{rgb}{1,0,0}
	\definecolor{colString}{rgb}{0,0.5,0}
	\definecolor{darkyellow}{rgb}{1,0.9,0}
	\definecolor{myblue}{RGB}{80,80,160}
	\definecolor{mygreen}{RGB}{80,160,80}
	\definecolor{shadecolor}{rgb}{0.87,0.87,0.87} 
	\def\xstrut{\vphantom{\dfrac{(A)^1}{(B)^1}}}
%\graphicspath{C:/Users/s0961/Desktop/自編講義}
\usepackage{qrcode}

% for drawing graphs
%\usepackage{tkz-fct,tikz-cd,pst-solides3d}
\usepackage{tikz,tkz-euclide,pst-eucl,pgf,pgfplots}
\usetikzlibrary{shapes,shapes.geometric}
\usetikzlibrary{decorations.markings,decorations.pathmorphing}
\usetikzlibrary{positioning,chains,fit,calc,intersections,tikzmark,through}
\usetikzlibrary{arrows,arrows.meta,automata,datavisualization,datavisualization.formats.functions}
\usetikzlibrary{graphs,graphs.standard,3d,perspective}

\tikzset{every picture/.style={thick}}
\tikzset{arrowMe/.style={postaction=decorate,decoration={markings, mark=at position .5 with {\arrow[thick]{#1}}}}}



\usepgfplotslibrary{polar}
\usepgflibrary{shapes.geometric}
\usepackage{forest}
\forestset{
  my tree/.style={
    before typesetting nodes={
      for tree={
        if={>O+tt=!O_=&{content}{}{n}{1}}{
          label/.process={Ow}{content}{left:##1},
        }{
          label/.process={Ow}{content}{right:##1},
       },
        content=,
      },
    },
    where level=0{
      draw,
      baseline,
      circle,
    }
    {
      draw,
      fill,
      circle,
    },
    for tree={inner sep=1.5pt, s sep'+=10pt},
  },
  default preamble={my tree},
  cut/.style={
    tikz+={
      \path [decorate, decoration={markings, mark=at position .5 with {\draw [] +(90:2.5pt) -- +(-90:2.5pt);}}] ()  -- (!u);
    }
  }
}
\thispagestyle{empty}
\pgfplotsset{my style/.append style={axis x line=middle, axis y line=
middle, xlabel={$x$}, ylabel={$y$}, axis equal }}

\newcommand{\mybox}[4][\textwidth-\pgfkeysvalueof{/pgf/inner xsep}-2mm]{%
\begin{figure}[!h]
\centering
\begin{tikzpicture}
\node[line width=.5mm, rounded corners, draw=#2, inner ysep=10pt, text width=#1, outer sep=0] (one) {\vspace*{15pt}\\\begin{varwidth}{\textwidth}#4\end{varwidth}};
\node[text=white,anchor=north east,align=center, minimum height=20pt] (two) at (one.north east) {#3 \hspace*{.5mm}};
\path[fill=#2]
    (one.north west|-two.west) --
    ($(two.west)+(-1.5cm,0)$)
    to[out=0,in=180] (two.south west) --
    (two.south east) [rounded corners] --
    (one.north east) --
    (one.north west) [sharp corners] -- cycle;
\node[text=white,anchor=north east,align=center, minimum height=20pt, text height=2ex] (three) at (one.north east) {#3 \hspace*{.5mm}};
\end{tikzpicture}
\end{figure}
}

\usepackage[most]{tcolorbox}
% user defined color
\usepackage{xcolor}
\definecolor{observecolor}{RGB}{153,201,227}
\definecolor{explorecolor}{RGB}{178,217,200}

\tcbset{
    common/.style={
		enhanced,
		arc=0mm,
		fonttitle=\large\bfseries,
		coltitle=black,
		attach boxed title to top left={xshift=0mm,
										yshift=-0.50mm},
		boxed title style={
			skin=enhancedfirst jigsaw,
			size=small,
			arc=5mm,
			bottom=0mm,
			left=8mm,
			right=18mm,
			top=1mm},
			boxrule=0pt,
			frame hidden},
	observestyle/.style={
		common,
		colbacktitle=observecolor,
		colframe=observecolor,
		colback=observecolor!40,
		borderline north={4pt}{0pt}{observecolor}}
}

\newtcolorbox{observe}{observestyle,title=小試身手}
\newtcolorbox{custom}[2][gray]{
	common,
	title=#2,
	colbacktitle=#1,
	colframe=#1,
	colback=#1!40,
	borderline north={4pt}{0pt}{#1}}


\usepackage[most]{tcolorbox}
% user defined color
\usepackage{xcolor}
\definecolor{observecolor}{RGB}{153,201,227}
\definecolor{explorecolor}{RGB}{178,217,200}

\tcbset{
    common/.style={
		enhanced,
		arc=0mm,
		fonttitle=\large\bfseries,
		coltitle=black,
		attach boxed title to top left={xshift=0mm,
										yshift=-0.50mm},
		boxed title style={
			skin=enhancedfirst jigsaw,
			size=small,
			arc=5mm,
			bottom=0mm,
			left=8mm,
			right=18mm,
			top=1mm},
			boxrule=0pt,
			frame hidden},
	observestyle/.style={
		common,
		colbacktitle=observecolor,
		colframe=observecolor,
		colback=observecolor!40,
		borderline north={4pt}{0pt}{observecolor}}
}

\newtcolorbox{observing}{observestyle,title=小試身手}
\newtcolorbox{customs}[2][brown]{
	common,
	title=#2,
	colbacktitle=#1,
	colframe=#1,
	colback=#1!40,
	borderline north={4pt}{0pt}{#1}}
\tcbuselibrary{skins,breakable,theorems}	

\makeatletter\long\def\ifnodedefined#1#2{\@ifundefined{pgf@sh@ns@#1}{}{#2}}\makeatother
\tikzset{
    %line-styles
        OO/.style={fill=blue,draw=blue,{Circle[width=1mm,length=1mm]}-{Circle[width=1mm,length=1mm]},shorten <= -0.5mm,shorten >= -0.5mm}, <O/.style={stealth-{Circle[width=1mm,length=1mm]},shorten <= 0mm}, O>/.style={{Circle[width=1mm,length=1mm]}-stealth,shorten >= 0mm}, <OO>/.style={stealth-stealth,shorten <= 0mm,shorten >= 0mm},every edge/.append style={OO},
    %how to plot:
        pics/.cd, plot inequality/.code n args={4}{
        %all the paths we need to figure out the intersections etc.
            \path[name path=conditionline] plot[variable=\x,domain={#2*1.01}:{#3*1.01},samples=300] ({\x},{ifthenelse(#1,{1+#4},{-1+#4})}); \path[name path=zeroline] (#2,#4) -- (#3,#4);\path[name path=start] (#2,{#4+1.1}) -- (#2,{#4+.9});\path[name path=end] (#3,{#4+1.1}) -- (#3,{#4+0.9});\path[name intersections={of=conditionline and start,name=startt}];\path[name intersections={of=conditionline and end,name=endd}];\path[name intersections={of=conditionline and zeroline,name=zerolinee}];\coordinate (intersection-0) at (#2,#4);
        %draw lines, nots and arrows
            \ifnum0\ifnodedefined{zerolinee-1}{1}\ifnodedefined{startt-1}{1}>0\draw[name intersections={of=zeroline and conditionline,total=\t,by=x}]\foreach \i [count=\s, evaluate=\s as \startswitch using iseven(\s+0\ifnodedefined{startt-1}{1})] in {0,...,{\t}}{\ifnum\i=\t coordinate (intersection-\s) at (#3,#4)\fi\if1\startswitch(intersection-\i) edge [\ifnodedefined{startt-1}{\ifnum\i=0<O\fi}\ifnodedefined{endd-1}{\ifnum\i=\t O>\fi}] (intersection-\s)\fi};\pgfnoderename{}{startt-1}\pgfnoderename{}{endd-1}\pgfnoderename{}{zerolinee-1}\fi}
}


% for margins
\usepackage[top=2.5cm,bottom=2.5cm,left=2cm,right=2cm,a4paper]{geometry}


% 表格高度及長內容換行
\newcommand{\xrowht}[2][0]{\addstackgap[.5\dimexpr#2\relax]{\vphantom{#1}}}
\newcommand{\tabincell}[2]{\begin{tabular}{@{}#1@{}}#2\end{tabular}}

%\setlength{\parindent}{2em}
% MACROS
\newenvironment{sol}{\medskip\noindent {\bf Solution.}}{\newpage}
\newcommand{\mystrut}{\rule[-.5\baselineskip]{0pt}{2\baselineskip}}
\newcommand{\divisible}{\mathrel{|}}
\newcommand{\bx}{{\bf x}}
\newcommand{\trans}{^\top}
\newcommand{\id}[1]{\ensuremath{\,\mathrm{d} #1}\xspace}
\newcommand{\eva}[1]{\ensuremath{\left.#1\right|}\xspace}
\newcommand{\ppair}[1]{\ensuremath{\left(#1\right)}\xspace}
\newcommand{\apair}[1]{\ensuremath{\left\langle#1\right\rangle}\xspace}
\newcommand{\bpair}[1]{\ensuremath{\left[#1\right]}\xspace}
\newcommand{\cpair}[1]{\ensuremath{\left\lceil #1\right\rceil}\xspace}
\newcommand{\fpair}[1]{\ensuremath{\left\lfloor #1\right\rfloor}\xspace}
\newcommand{\Bpair}[1]{\ensuremath{\left\{#1\right\}}\xspace} 
\newcommand{\vpair}[1]{\ensuremath{\left|#1\right|}\xspace}
\newcommand{\bmat}[1]{\ensuremath{\begin{bmatrix}#1\end{bmatrix}}\xspace} 
\newcommand{\Bmat}[1]{\ensuremath{\begin{Bmatrix}#1\end{Bmatrix}}\xspace} 
\newcommand{\vmat}[1]{\ensuremath{\begin{vmatrix}#1\end{vmatrix}}\xspace}
\newcommand{\Vmat}[1]{\ensuremath{\begin{Vmatrix}#1\end{Vmatrix}}\xspace} 
\newcommand{\pmat}[1]{\ensuremath{\begin{pmatrix}#1\end{pmatrix}}\xspace}
\renewcommand{\multirowsetup}{\centering} %合併表格後, 內容水平且垂直置中
%\renewcommand\arraystretch{1.8}%調整表格內行高
\newlength{\iconwidth} %定义ico标识的宽度
\setlength{\iconwidth}{1cm} %宽度赋值
\definecolor{boxheadcol}{gray}{.9} %定义所需的颜色,需要xcolor支持。
\definecolor{boxcol}{gray}{.5} %同上
\setlength\cellspacetoplimit{3pt}
\setlength\cellspacebottomlimit{3pt}
\setcellgapes{3pt}

%選擇題
\newcommand{\fourch}[4]{%~\hfill(\qquad)\\
\begin{tabular}{*{4}{@{}p{0.25\textwidth}}}(A)~#1 & (B)~#2 & (C)~#3 & (D)~#4\end{tabular}}
\newcommand{\twoch}[4]{%~\hfill(\qquad)\\
\begin{tabular}{*{2}{@{}p{0.5\textwidth}}}(A)~#1 & (B)~#2\end{tabular}\\\begin{tabular}{*{2}{@{}p{0.5\textwidth}}}(C)~#3 & (D)~#4\end{tabular}}
\newcommand{\onech}[4]{%~\hfill(\qquad)\\
(A)~#1 \\ (B)~#2 \\ (C)~#3 \\ (D)~#4}
\newlength\widthcha
\newlength\widthchb
\newlength\widthchc
\newlength\widthchd
\newlength\widthch
\newlength\tabmaxwidth
\setlength\tabmaxwidth{1\textwidth}
\newlength\fourthtabwidth
\setlength\fourthtabwidth{0.25\textwidth}
\newlength\halftabwidth
\setlength\halftabwidth{0.5\textwidth}

\newcommand{\choice}[4]{\settowidth\widthcha{AM.#1}\setlength{\widthch}{\widthcha}
    \settowidth\widthchb{BM.#2}
    \ifthenelse{\widthch<\widthchb}{\setlength{\widthch}{\widthchb}}{}
    \settowidth\widthchb{CM.#3}
    \ifthenelse{\widthch<\widthchb}{\setlength{\widthch}{\widthchb}}{}
    \settowidth\widthchb{DM.#4}
    \ifthenelse{\widthch<\widthchb}{\setlength{\widthch}{\widthchb}}{}
    \ifthenelse{\widthch<\fourthtabwidth}{\fourch{#1}{#2}{#3}{#4}}
    {\ifthenelse{\widthch<\halftabwidth\and\widthch>\fourthtabwidth}{\twoch{#1}{#2}{#3}{#4}}
        {\onech{#1}{#2}{#3}{#4}}}}

\everymath{\displaystyle} %所有數學模式都用展式數學模式
%填充題
%有計數的填充
%\newcounter{blanknumbering} % running number for filling blank 
%\def\blank{\addtocounter{blanknumbering}{1}
%\ensuremath{\underline{\hspace*{.5cm}\textcircled{\theblanknumbering}\hspace*{.5cm}}}}

%填充題
\def\blank{\ensuremath{\underline{\hspace*{1cm}}}}

%短除法
\newcounter{divline}
\def\rlwd{.5pt} \def\rlht{\dimexpr\dp\strutbox+\ht\strutbox} \def\rldp{.75ex}
\newcommand\mydiv[3][\relax]{%
  \ifx\relax#1\stepcounter{divline}\else\setcounter{divline}{#1}\fi%
  \mbox{}\hspace{\thedivline\dimexpr1ex}#2~\setbox0=\hbox{~$#3$}%
  \dumbstackengine{-\rlwd}{\rule[-\rldp]{\rlwd}{\rlht}~#3}{\rule{\dimexpr4pt+\wd0}{\rlwd}}%
}
\def\remainder#1{\stepcounter{divline}%
  \mbox{}\hspace{\dimexpr1ex+\thedivline\dimexpr1ex}~#1\setcounter{divline}{0}}
\makeatletter
\global\newlength\@stackedboxwidth
\newlength\@boxshift
\newsavebox\@addedbox
\newsavebox\@anchorbox
\newcommand*\dumbstackengine[3]{%
    \sbox{\@anchorbox}{$#2$}%
    \sbox{\@addedbox}{$#3$}%
    \setlength{\@stackedboxwidth}{\wd\@anchorbox}%
      \ifdim\wd\@addedbox>\@stackedboxwidth%
        \setlength{\@stackedboxwidth}{\wd\@addedbox}%
      \fi%
        \setlength{\@boxshift}{\dimexpr-\dp\@anchorbox -\ht\@addedbox -#1}%
        \usebox{\@anchorbox}%
        \hspace{-\wd\@anchorbox}%
        \raisebox{\@boxshift}{\usebox{\@addedbox}}%
        \hspace{-\wd\@addedbox}%
        \hspace{\@stackedboxwidth}%
}

\newtcbtheorem{question}{範例}%
  {enhanced, breakable,
    colback = white, colframe = cyan, colbacktitle = cyan,
    attach boxed title to top left = {yshift = -2mm, xshift = 5mm},
    boxed title style = {sharp corners},
    fonttitle = \sffamily\bfseries, separator sign = {~}}{qst}

% 直線表示
%\def\shrinkage{-2.4mu}
%\def\vecsign#1{\rule[1.388\LMex]{\dimexpr#1-2.5pt}{.36\LMpt}%
%  \kern-6.0\LMpt\mathchar"017E}
%\def\dvecsign#1{\rule{0pt}{7\LMpt}\smash{\stackon[-1.989\LMpt]{%
%  \SavedStyle\mkern-\shrinkage\vecsign{#1}}%
%  {\rotatebox{180}{$\SavedStyle\mkern-\shrinkage\vecsign{#1}$}}}}
%\def\dvec#1{\ThisStyle{\setbox0=\hbox{$\SavedStyle#1$}%
%  \def\useanchorwidth{T}\stackon[-4.2\LMpt]{\SavedStyle#1}{\,\dvecsign{\wd0}}}}
%\stackMath